
\section{Detailed research plan}\label{detailed-research-plan}

The GenomeSwift pipeline is designed to enhance genomic analysis through
automation and integration of various analytical tools. Our detailed
research plan outlines the methodology and statistical approaches we
will employ to ensure the effectiveness and efficiency of GenomeSwift in
processing and analysing genomic data.

\textbf{(i) Integration of existing tools:} The pipeline will integrate
several tools that we have developed, including:
\href{https://github.com/DylanLawless/ProteoMCLustR}{ProteoMCLustR} for
protein pathway clustering {[}7{]}, {[}8{]}; SkatRbrain for statistical
analysis of genetic data {[}9{]}, {[}10{]}, {[}11{]};
\href{https://github.com/DylanLawless/archipelago}{Archipelago} for a
unified representation for genomic statistical analysis;
\href{https://github.com/DylanLawless/ACMGuru}{ACMGuru} for clinical
genetic interpretation {[}12{]}; AutoConstructR for protein structure
plotting, facilitating a comprehensive interpretation of genetic
variations {[}16{]}; and other modular tools.

\textbf{(ii) Data processing and analysis workflow: Data input}:
GenomeSwift will accept raw genomic data, applying preprocessing steps
to ensure data quality and compatibility. \textbf{Variant detection}:
Utilising the best practices from tools like GATK, the pipeline will
perform variant calling, ensuring high-confidence identification of
genetic variations {[}4{]}. \textbf{Statistical analysis}:Employing
SkatRbrain, the pipeline will conduct robust statistical analyses to
associate genomic variations with disease phenotypes, including rare
variant analysis {[}9{]}, {[}10{]}, {[}11{]}. \textbf{Clinical
interpretation}: ACMGuru will be used to interpret the clinical
significance of detected variants, aligning with the American College of
Medical Genetics guidelines {[}12{]}.

\textbf{(iii) Simulation and validation:} The pipeline\textquotesingle s
efficacy will be validated using simulated datasets encompassing various
disease scenarios (rare variant, common variant, polygenic risk) to
ensure its robustness across different genetic contexts {[}3{]},
{[}17{]}, {[}18{]}. Validation will also include real-world data from
Swiss hospitals to confirm the pipeline\textquotesingle s practical
applicability and accuracy.

\textbf{(iv) Statistical methodologies}: GenomeSwift will incorporate
advanced statistical methods to analyse the association between genetic
variants and diseases, ensuring the analyses are powered adequately to
detect significant associations even in the context of rare diseases.
The pipeline will employ a range of statistical tests suitable for
different data types and study designs, ensuring the flexibility and
comprehensiveness of the analysis. Specifically, optimised sequence
kernel association tests (SKAT-O) will form the basis of statistical
validation tests {[}10{]}. Successful outcomes will therefore
demonstrate the ability to substitute compatible drop-in methods; burden
tests such as CMC {[}19{]} and WSS {[}20{]}, variance component tests
such as C-alpha {[}21{]} and SKAT {[}9{]}, combined burden and variance
component tests such as SKAT-O {[}10{]}, other combination tests such as
ACAT-RVAT {[}22{]}, regression and generalised mixed models such as
REGENIE {[}23{]} and SAGE-GENE+ {[}24{]}, and others.

\textbf{(v) Automation and user interface}: The pipeline will feature
containerisation to support development and use. Automation will be a
key focus, with the pipeline designed to require minimal user
intervention, streamlining the analysis process from start to finish.
User output will include graphical interfaces and technical reporting
documents.

\textbf{(vi) Output and reporting}: GenomeSwift will generate
comprehensive reports, detailing the analysis results, including variant
identification, statistical associations, and clinical interpretations.
The pipeline will ensure that outputs are presented in an easily
interpretable format, facilitating clinical decision-making and further
research. The key technical data will also be generated including
formats for reporting with SPHN RDF schema concepts.

\hypertarget{statement-of-the-relevancesignificance-of-the-project}{%
\section{Statement of the relevance/significance of the
project}\label{statement-of-the-relevancesignificance-of-the-project}}

GenomeSwift represents a pivotal development at the intersection of
advanced genomics and clinical application, poised to influence
diagnosis and treatment protocols. Its significance spans several
critical aspects for \kispi:

- \textbf{Healthcare Impact:} GenomeSwift will enhance the diagnostic
process for rare diseases by delivering rapid and precise genomic
analysis. This capability enables more timely and \textbf{accurate
treatment decisions}, crucial for improving patient outcomes. By
providing detailed genetic insights, GenomeSwift supports the shift
towards \textbf{personalised medicine}, where treatments are
specifically tailored to individual genetic profiles.

\textbf{- Strategic alignment:} GenomeSwift aligns with the strategic
objectives of the Children's Hospital Zurich, notably enhancing the
institution\textquotesingle s \textbf{capacity} for genetic diagnostics
and personalised healthcare. The project also contributes to the
efficiency and effectiveness of the \textbf{broader Swiss healthcare
system}, underscoring innovation and leadership in medical genomics.

\textbf{- Scientific advancement:} This project marks a considerable
advance by amalgamating multiple analytical tools into a unified
pipeline, thereby establishing a new benchmark in genomic analysis.
GenomeSwift will streamline genomic data processing,
\textbf{accelerating} research into rare diseases and potentially
revealing novel therapeutic targets.

\textbf{- Collaboration and Education:} GenomeSwift will enhance
\textbf{collaboration} across researchers, clinicians, and institutions
by offering a shared genomic analysis format, fostering an integrated
healthcare and research approach. Additionally, it will serve as an
\textbf{educational resource} to improve genomic literacy and train
future experts, clarifying the application of clinical genetics
guidelines in analysis for healthcare professionals.

\textbf{- Sustainability and accessibility}: As an open-source
initiative, GenomeSwift is accessible to a diverse user base, promoting
ongoing support. The project\textquotesingle s focus on understanding
and treating diseases has the potential to make a \textbf{lasting
impact} on healthcare and research, leading to enhanced patient outcomes
and broader scientific insights.
