\section{Summary}

\textbf{Background}: The integration of advanced genomic technologies into healthcare, particularly for rare genetic diseases, is crucial for improving diagnostic and treatment strategies. The current fragmented analysis methodologies delay critical medical responses. Our proposal aims to streamline and automate these processes. This can provide a new opportunity for precision medicine in Kispi by offering rapid, precise insights into genetic disorders.

\textbf{Objective}: The primary goal of GenomeSwift is to develop and deploy a comprehensive, automated software pipeline that enhances the speed and accuracy of genomic data analysis, particularly for diagnosing and treating genetic diseases in Kispi.

\textbf{Research Plan}: GenomeSwift will integrate several purpose-build tools (ProteoMCLustR, SkatRbrain, Archipelago, ACMGuru, and DeepInferR), along with several industry-standards, into one unified platform to optimise genomic data processing from initial input to clinical interpretation. The project will employ robust statistical methodologies and simulation/validation processes to ensure reliability and clinical relevance.

\textbf{Relevance/Significance}: By enhancing genomic analysis capabilities, GenomeSwift aligns with Kispi’s strategic goals and addresses an unmet need in personalised medicine. It promises to reduce the time to diagnosis and treatment, significantly impacting patient outcomes, particularly for children with rare genetic disorders.

\textbf{Personnel}: The project will be led by Dylan Lawless, a specialist in genomics and bioinformatics, supported by a multidisciplinary team including Prof. Luregn Schlapbach, Prof. Jacques Fellay, and other experts from UZH, ETHZ, EPFL, and CHUV.
Collaborations: National and international collaborations with institutions like the SwissMultiOmic Center, CHUV, EPFL, and the Global Alliance for Genomics and Health will enrich the project, ensuring a broad and effective implementation.

\textbf{Timetable and Budget}: A detailed timetable using CPM and PERT methods predicts successful project completion within 53 weeks. The budget of CHF 98’076 covers personal and material costs, supported by existing infrastructure and collaborative resources.