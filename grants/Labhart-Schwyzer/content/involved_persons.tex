
\section{Collaborations and network}
\subsection{Involved persons}

The GenomeSwift project is supported by a multidisciplinary team of
experts, each bringing unique expertise and experience to ensure the
project\textquotesingle s success. \textbf{Prof. Luregn Schlapbach} will
act as \textbf{mentor}: project management, expected milestones, and
outcomes will be tracked with progress reports. \textbf{Prof. Jacques
Fellay} will provide \textbf{advice} on the tool requirements, essential
guidance on data privacy, consent, and ethical considerations, ensuring
that GenomeSwift adheres to the highest standards of research ethics and
data protection. \textbf{Dr. Dylan Lawless} will act as \textbf{project
leader}: established in genomics and bioinformatics, with extensive
experience in developing computational tools for genomic analysis. He
has expertise in the understanding genetic variations and their
implications in diseases, especially in the context of rare disease in
children. \textbf{Dr. Vito Zanotelli} will act as bioinformatics specialist: he has a
profound understanding of integrating and analysing large-scale genomic
datasets. His expertise is crucial in refining the data processing
algorithms and ensuring that GenomeSwift can handle complex and
voluminous datasets efficiently. 
\textbf{Ali Saadat} has a background in genetic analysis method development, bringing valuable insights into the
clinical implications of genomic findings. His focus is on translating genomic data into actionable clinical knowledge, which is instrumental in designing the interpretative aspects of GenomeSwift. 
\textbf{Dr Zhi Ming Xu} will advise as genomic analysis statistics and software development. A \textbf{PhD student} will be employed to work on development of novel
research features.

Additional collaborating bioinformatics specialists from UZH, ETHZ,
EPFL, and CHUV will integrate and optimise the computational tools
within GenomeSwift. They will test the flow of data in the format
matching our data providers (1) (e.g.,
\href{http://smoc.ethz.ch/}{SwissMultiOmic Center}) to (2) HPC clusters
(e.g., BioMedIT) into (3) GenomeSwift. They will test the clinical
applicability and relevance of the pipeline. Open-source development
will be tested by our collaborators in
\href{https://www.swisspedhealth.ch/}{swisspedhealth.ch}, and
\href{https://www.epfl.ch/labs/fellay-lab/}{EPFL}. Software will be
tested on multiple nodes including ETHZ
\href{https://unlimited.ethz.ch/display/LeoMed2/Leonhard+Med+Intro+for+shareholders}{SIS
Leonhard Med} and University of Basel
\href{https://scicore.unibas.ch/}{sciCORE Med}.


\subsection{National / international collaborations}

GenomeSwift is enhanced by an extensive network of collaborations both
nationally and internationally, significantly impacting healthcare and
research communities. These partnerships facilitate knowledge sharing
and resource exchange, crucial for the project\textquotesingle s
success. \textbf{SwissMultiOmic Center:} Serving as our primary data
provider, this key national partner offers access to state-of-the-art
technologies and datasets, enabling GenomeSwift to utilise high-quality
genomic data and analytical resources. Regular communication between our
groups aids in refining and advancing the pipeline. \textbf{Centre
Hospitalier Universitaire Vaudois (CHUV):} Collaboration with CHUV
researchers to review GenomeSwift\textquotesingle s adaptability across
different healthcare settings, ensuring its effectiveness and
versatility. CHUV's commitment to medical genetics research provides a
solid basis for collaborative enhancements of the pipeline.
\textbf{Ecole Polytechnique Fédérale de Lausanne (EPFL) and ETH Zurich:}
These partnerships grant GenomeSwift access to extensive expertise in
bioinformatics, computational biology, and genomics, fostering a
multidisciplinary integration of varied user needs and methodologies.
\textbf{Global Alliance for Genomics and Health (GA4GH):} GenomeSwift
aligns with GA4GH standards to enhance data interoperability and
security worldwide (\url{https://www.ga4gh.org}). This partnership
ensures GenomeSwift\textquotesingle s integration into global genomic
databases, adhering to international best practices in data privacy and
ethics, thereby contributing to the global \textquotesingle internet of
genomics\textquotesingle. \textbf{Open-source software community:} As a
collaborative open-source project, GenomeSwift benefits from the
collective insights of a diverse group of developers and researchers,
promoting continuous innovation and ensuring the
platform\textquotesingle s ongoing availability and improvement.