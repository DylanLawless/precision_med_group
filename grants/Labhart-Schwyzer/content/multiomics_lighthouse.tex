\subsection{Multi-omics for diagnosis and discovery to inborn errors of metabolism}

We have previously demonstrated the potential for our approach through the SwissPedHealth lighthouse project (\url{https://www.swisspedhealth.ch}).
This stands out as a vanguard in leveraging multi-omics technologies for the diagnosis and discovery of rare metabolic diseases. 
This initiative has successfully applied genomic, transcriptomic, proteomic, and metabolomic analyses to provide a profound impact on understanding and treating disorders such as Methylmalonic Aciduria (MMA).

\subsubsection*{Detailed Findings from Phase 1: Methylmalonic Aciduria Study}
\begin{itemize}
    \item \textbf{Disease Focus and Genetic Insights:} Focusing on MMA, an inborn error of metabolism, the project identified pathogenic variants in the methylmalonyl-CoA mutase (MMUT) gene in 84\% of the studied cases (177 out of 210). This highlights the genetic heterogeneity and complex clinical presentations associated with MMA.
    \item \textbf{Advanced Multi-Omics Methods:} The study utilized a comprehensive suite of multi-omics approaches, integrating whole-genome sequencing (WGS), RNA sequencing (RNA-seq), proteotyping via data-independent acquisition mass spectrometry (DIA–MS), and detailed metabolomic analyses. This integration allowed for a nuanced understanding of the disease at multiple biological levels.
    \item \textbf{Metabolic Pathway Disruptions:} Significant disruptions were discovered in the tricarboxylic acid (TCA) cycle and its anaplerosis, primarily involving glutamine. These disruptions were extensively characterized through multi-organ metabolomics and stable-isotope tracing in a hemizygous Mmut mouse model, providing a clear pathophysiological pathway that contributes to the disease state.
    \item \textbf{Protein Interactions and Therapeutic Insights:} The study underscored crucial interactions between MMUT and key enzymes such as glutamate dehydrogenase and oxoglutarate dehydrogenase. Treatment with dimethyl-oxoglutarate was shown to restore TCA cycle functionality, offering a novel therapeutic avenue that could significantly impact clinical outcomes for patients with MMA.
\end{itemize}

\subsubsection*{Ongoing and Future Research Phases}
\begin{itemize}
    \item \textbf{Phase 2 and Phase 3:} Building on the findings from Phase 1, ongoing phases aim to extend these multi-omics methodologies to broader cohorts with extreme phenotypes. The project seeks to refine diagnostic workflows and explore the real-time clinical application of these findings in prospective studies involving children with severe metabolic dysfunctions.
\end{itemize}


%\subsubsection*{Conclusion}
%The integration of multi-omics data within the SwissPedHealth Lighthouse Project exemplifies a groundbreaking approach to rare metabolic disease research. The findings from the MMA study particularly underscore the potential of such integrated approaches to revolutionize diagnostics and treatment, ultimately enhancing patient care in pediatric settings.
