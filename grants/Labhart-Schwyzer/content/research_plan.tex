
\section{Materials and methods}\label{detailed-research-plan}

The GenomeSwift pipeline is designed to enhance genomic analysis through
automation and integration of various analytical tools. Our detailed
research plan outlines the methodology and statistical approaches we
will employ to ensure the effectiveness and efficiency of GenomeSwift in
processing and analysing genomic data.

\textbf{(i) Integration of existing tools:} The pipeline will integrate
several tools that we have developed, including:
\href{https://github.com/DylanLawless/ProteoMCLustR}{ProteoMCLustR} for
protein pathway clustering \citep{ref7}, \citep{ref8}; SkatRbrain for statistical
analysis of genetic data \citep{ref9}, \citep{ref10}, \citep{ref11};
\href{https://github.com/DylanLawless/archipelago}{Archipelago} for a
unified representation for genomic statistical analysis;
\href{https://github.com/DylanLawless/ACMGuru}{ACMGuru} for clinical
genetic interpretation \citep{ref12}; AutoConstructR for protein structure
plotting, facilitating a comprehensive interpretation of genetic
variations \citep{ref16}; and other modular tools.

\textbf{(ii) Data processing and analysis workflow: Data input}:
GenomeSwift will accept raw genomic data, applying preprocessing steps
to ensure data quality and compatibility. \textbf{Variant detection}:
Utilising the best practices from tools like GATK, the pipeline will
perform variant calling, ensuring high-confidence identification of
genetic variations \citep{ref4}. \textbf{Statistical analysis}:Employing
SkatRbrain, the pipeline will conduct robust statistical analyses to
associate genomic variations with disease phenotypes, including rare
variant analysis \citep{ref9}, \citep{ref10}, \citep{ref11}. \textbf{Clinical
interpretation}: ACMGuru will be used to interpret the clinical
significance of detected variants, aligning with the American College of
Medical Genetics guidelines \citep{ref12}.

\textbf{(iii) Simulation and validation:} The pipeline's
efficacy will be validated using simulated datasets encompassing various
disease scenarios (rare variant, common variant, polygenic risk) to
ensure its robustness across different genetic contexts \citep{ref3},
\citep{ref17}, \citep{ref18}. Validation will also include real-world data from
Swiss hospitals to confirm the pipeline's practical
applicability and accuracy.

\textbf{(iv) Statistical methodologies}: GenomeSwift will incorporate
advanced statistical methods to analyse the association between genetic
variants and diseases, ensuring the analyses are powered adequately to
detect significant associations even in the context of rare diseases.
The pipeline will employ a range of statistical tests suitable for
different data types and study designs, ensuring the flexibility and
comprehensiveness of the analysis. Specifically, optimised sequence
kernel association tests (SKAT-O) will form the basis of statistical
validation tests \citep{ref10}. Successful outcomes will therefore
demonstrate the ability to substitute compatible drop-in methods; burden
tests such as CMC \citep{ref19} and WSS \citep{ref20}, variance component tests
such as C-alpha \citep{ref21} and SKAT \citep{ref9}, combined burden and variance
component tests such as SKAT-O \citep{ref10}, other combination tests such as
ACAT-RVAT \citep{ref22}, regression and generalised mixed models such as
REGENIE \citep{ref23} and SAGE-GENE+ \citep{ref24}, and others.

\textbf{(v) Automation and user interface}: The pipeline will feature
containerisation to support development and use. Automation will be a
key focus, with the pipeline designed to require minimal user
intervention, streamlining the analysis process from start to finish.
User output will include graphical interfaces and technical reporting
documents.

\textbf{(vi) Output and reporting}: GenomeSwift will generate
comprehensive reports, detailing the analysis results, including variant
identification, statistical associations, and clinical interpretations.
The pipeline will ensure that outputs are presented in an easily
interpretable format, facilitating clinical decision-making and further
research. The key technical data will also be generated including
formats for reporting with SPHN RDF schema concepts.
