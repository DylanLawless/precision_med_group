
\section{Objective and hypotheses}\label{hypotheses-and-objective}

The primary objective of this study is to develop and implement
GenomeSwift, an automated, comprehensive software pipeline designed for
rapid, precise genomic data analysis, with a particular focus on the
diagnosis and treatment of genetic diseases in \kispi.

\hypertarget{hypotheses}{%
\subsection{Hypotheses:}\label{hypotheses}}

\textbf{(i) Automation and integration efficiency}: Automating the
genome analysis process and integrating various existing tools into a
single pipeline, GenomeSwift is expected to significantly expedite
genomic data processing, thus enhancing the efficiency and speed of
diagnosing and planning treatments for diseases. \textbf{(ii) Enhanced
diagnostic accuracy}: By incorporating advanced tools for variant
detection, statistical analysis, and clinical interpretation,
GenomeSwift aims to improve diagnostic accuracy for rare diseases,
leading to more precise and personalised treatment approaches.
\textbf{(iii) Impact on disease research}: The use of GenomeSwift in
rare disease contexts is anticipated to not only improve clinical
outcomes but also to enrich genomic research, offering a workflow that
provides traceable genetic evidence to better classify variants.

\hypertarget{objectives}{%
\subsection{Objectives:}\label{objectives}}

\textbf{(i) Tool integration} Integrate existing tools such as
ProteoMCLustR, SkatRbrain, and ACMGuru into a unified and automated
pipeline to streamline the process from data input to clinical
interpretation. \textbf{(ii) Validation and refinement}: Validate
GenomeSwift using both simulated {[}13{]} and real-world datasets to
confirm its reliability and accuracy across various clinical scenarios,
refining the pipeline based on performance metrics and feedback
{[}14{]}, {[}15{]}. \textbf{(iii) User accessibility}: Ensure
GenomeSwift is user-friendly for analysts while making evidence-based
results accessible to healthcare professionals and researchers,
facilitating broader adoption within \kispi and potentially other
institutions. \textbf{(iv) Knowledge dissemination}: Disseminate the
findings and capabilities of GenomeSwift through publications and
collaborations, aiming to enhance the use of genomic data for healthcare
improvement and scientific discovery.

Figure . From DNA to diagnosis: Variant effect evidence is assessed
based on standardised guidelines. Data is produced in an
analysis-friendly formats for statistical or AI/ML reuse. GenomeSwift
produces clinical genetics reports and database results using SPHN RDF
schema concepts.
