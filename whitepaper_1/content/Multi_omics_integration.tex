
\section{Multi-omics integration}

Our unit will use whole genome sequencing (WGS) to identify both common and rare genetic variants, thereby establishing a foundational genetic blueprint of disease predisposition. In complement to WGS, RNA sequencing (RNAseq) provides insights into gene expression changes across different physiological states or in response to treatments, shedding light on the functional consequences of genetic alterations.

Furthermore, the incorporation of proteomics and metabolomics technologies offers a detailed examination of the proteome and metabolome, which are in closer correspondence to the phenotype. These methodologies facilitate the monitoring of biochemical activities and protein functions that have direct impacts on disease phenotypes, effectively bridging genotypic information with tangible biological and clinical outcomes.

In addition, routine clinical measurements and electronic health records (EHRs) will be integrated with omics data, augmenting the granularity and practicality of our findings. This synthesis enables tailored, real-time therapeutic interventions and continuous monitoring of disease progression, markedly enhancing the precision of diagnostics and personalised treatment plans.
