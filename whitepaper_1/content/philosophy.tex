\section*{Core philosophy}
At the \pmu of \kispi, we redefine translational medicine not merely through implementing cutting-edge methods but through a foundational philosophy that optimises our processes.


\subsection*{The challenge}
Our team used to spend extensive time on inefficient collaboration, searching for critical data such as patient statistics or file locations.
This redundancy slows progress and dilutes focus.

\subsection*{Our approach}
\begin{enumerate}
\item  Unified structure: We employ a consistent framework for all projects under the \pmu umbrella, streamlining setup and integration.
\item Single source of variables: All project-related data is tracked via a centralised system, eliminating repetitive searches.
\item Standardised format: A uniform format across the board ensures quick access to necessary information without confusion.
\end{enumerate}

\subsection*{The impact}
These strategies allow for seamless updates across projects when new data is added, mirroring the efficiency of a robust digital codebase 
\citep{potvin_levenberg_google_ACM}. 
This system not only enhances operational efficiency but also deepens our translational medicine capabilities.

\subsection*{Philosophy before tools}
We prioritise our strategic approach over specific tools. 
Our commitment to a single-source, everything-as-code philosophy demands the use of tools that support shared variables and robust data integration, avoiding traditional tools like MS Word and Excel in favour of industry-standard and preferably open-source options like Markdown, LaTeX, R, python, etc.
This also means that core data is secure and safe while the source code for all software and documents can be safely made public.

This philosophy not only streamlines our internal processes but also aligns closely with our translational medicine efforts, enabling us to apply innovative clinical methods more efficiently and effectively.
