%\section{Key performance indicators}
%
%To objectively assess the efficacy of our research endeavours and clinical implementations, the Precision Medicine Unit employs a comprehensive array of Key Performance Indicators (KPIs). These indicators encompass metrics such as data throughput, sample processing turnaround times, error rates in data analysis, user satisfaction among clinical staff, adherence to regulatory standards, operational cost efficiency, and the adoption of innovative tools and methods devised by our team.
%
%Regular evaluations of these KPIs enable us to monitor our performance meticulously and enact necessary modifications to enhance both efficiency and effectiveness. For instance, metrics like data throughput and sample processing times are pivotal in ensuring that genetic and omic analyses are conducted expeditiously, which is crucial for the timely diagnosis and treatment of paediatric patients. Error rates furnish insights into the precision of our analyses, directly impacting diagnostic reliability and patient safety.
%
%Additionally, feedback from clinical staff and other end-users is indispensable for refining our tools and methodologies. User satisfaction ratings assist us in evaluating the clinical relevance and ease of use of our analytical platforms. Compliance levels demonstrate our conformity to ethical standards and regulatory mandates, ensuring that our research activities comply with all pertinent guidelines and legislation.
%
%By adhering to stringent standards and actively seeking feedback, the Precision Medicine Unit not only maintains the highest levels of scientific integrity but also ensures that its contributions deliver significant benefits to paediatric healthcare. This continual process of evaluation and enhancement is essential for sustaining the confidence of our stakeholders and for the enduring success of our precision medicine initiatives.

\section{Key performance indicators}

To ensure that our research and clinical implementations are both effective and efficient, the \pmu will track performance using a well-defined set of Key Performance Indicators (KPIs). 
These KPIs are carefully chosen to reflect crucial aspects of our operations, from data management to user engagement and compliance.
We also aim to take the production burden off individuals and instead place it on the joint group effort to determine our own pace, remaining flexible to for worker satisfaction and development.

Our KPI framework includes the following metrics, each designed to provide insights into specific operational aspects:

\begin{itemize}
    \item \textbf{Data Throughput:} Measured by the total gigabytes (GB) of data processed weekly, this indicator helps assess our capacity to handle large-scale genomic datasets.
    \item \textbf{Turnaround Time:} This KPI tracks the duration from sample receipt to the delivery of the report, crucial for evaluating our efficiency in processing and reporting.
    \item \textbf{Research Output:} Annually quantified by the number of papers published and tools developed, reflecting our contribution to scientific knowledge and tool innovation.
    \item \textbf{Error Rate:} This is calculated as the number of errors per 100 analyses, providing a clear measure of the precision and reliability of our data analyses.
    \item \textbf{User Satisfaction:} Determined by averaging scores from user surveys, this metric gauges the satisfaction of clinical staff and researchers with our tools and reports.
    \item \textbf{Compliance:} Measured by the percentage of successful compliance checks, this KPI ensures that our operations adhere to all relevant regulatory and ethical standards.
    \item \textbf{Cost Efficiency:} This is evaluated by the cost per analysis or report, helping us monitor financial efficiency in our operations.
    \item \textbf{Training and Uptake:} The percentage of cases using new tools measures the adoption rate of our latest methodologies and technologies, indicating the effectiveness of our training programs.
\end{itemize}

